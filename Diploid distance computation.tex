\documentclass[a4paper]{article}

\usepackage[english]{babel}
\usepackage[utf8]{inputenc}
\usepackage{amsmath}
\usepackage{graphicx}
\usepackage[colorinlistoftodos]{todonotes}
\setlength\parindent{0pt}

\title{Steps of calculating proportion and distance with diploids}

\author{
  Duwal Shrestha, Mahendra\\
  \texttt{mduwalsh@utk.edu}
  }

\date{19 April 2015}


\begin{document}
\maketitle

%% Calculating proportion for infinite population %%
\section*{Calculation of proportion for infinite population}
\begin{flushleft}

Calculation of proportion for haploid is already done and we can use that q$^1_2$ code to calculate proportion for diploid since q$^{n+1}_{<\gamma0,\gamma1>}$ = p$^{n+1}_{<\gamma0,\gamma1>}$ p$^{n+1}_{<\gamma0,\gamma1>}$. 
Write a function q($\gamma0 ,\gamma1$) such that it takes diploid gamete as 2 long integer of certain bits and then returns proportion of the diploid gamete.

\end{flushleft}

\section*{Initial population proportion}
\begin{flushleft}
Initial population is randomly generated. However to maintain same proportion between infinite and finite population, proportion for all diploid gamete is randomly generated and stored in external file.
\end{flushleft}

\section*{Uniform crossover probability}
\begin{flushleft}
Crossover probability is defined by:
\linebreak
\begin{equation}
  \chi_i =\begin{cases}
    \chi2^{-l} & \text{if $i<0$}.\\
    1 - \chi + \chi2^{-l} & \text{if i = 0}.
  \end{cases}
\end{equation}
where i = 2$^l$ - 1

\end{flushleft}

\section*{Mutation}
\begin{flushleft}
\end{flushleft}

\section*{Reproducing new offsprings in finite population}
\begin{flushleft}
Choose two random parents for mating. 
\linebreak
Diploid chromosome of each parent go into crossover and mutation separately and two haploids from each parent are formed.
\linebreak
Haploid from one parent mates with haploid from other parent forming diploid resulting two diploids (offsprings).

\end{flushleft}

\section*{Distance computation}
\begin{flushleft}
Let $S$ be set of samples in finite population, let $f$ represent the
finite diploid population, and let $q$ represent the infinite diploid
population.  The squared distance between the finite and infinite
population is:
\begin{eqnarray}
\|f - q\|^2  & = &
\sum_{x} (f(x)-q(x))^2 \nonumber \\
& = &
\sum_{x \in S} (f(x)-q(x))^2 + \sum_{x \notin S} (f(x)-q(x))^2
\end{eqnarray}
In the second term of (2), $f(x) = 0$, therefore it is
\begin{eqnarray}
\sum_{x\notin S} q(x)^2 & = & \sum_{\langle x_1, x_2 \rangle\notin S}(p(x_1)p(x))^2 \nonumber \\
& = &
\sum_{\langle x_1, x_2 \rangle} p(x_1)^2p(x_2)^2  - \sum_{\langle x_1, x_2 \rangle\in S}  \big( p(x_1) p(x_2) \big)^2 \nonumber \\
& = &
\Big( \sum_{x_1} p(x_1)^2 \Big)^2 - \sum_{\langle x_1, x_2 \rangle\in S} \big( p(x_1) p(x_2) \big)^2
\end{eqnarray}

\texttt{Note:} Verify equation (3) is correct by checking it against
distance computed using equation (2) for smaller size of bits in
chromosome.




\end{flushleft}


\end{document}
